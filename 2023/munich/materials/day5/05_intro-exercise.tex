\documentclass[aspectratio=169]{beamer}
% \setbeamertemplate{footline}[frame number]
\usepackage{color,amsmath}
\usepackage{subfigure}
\usepackage{booktabs}
\usepackage{framed}
\usepackage{comment}
\hypersetup{
colorlinks=true,
linkcolor=blue,
filecolor=magenta,      
urlcolor=blue,
pdftitle={Overleaf Example},
pdfpagemode=FullScreen}
\usepackage{tabularx}

%%%%%%%%%%%%%%%%%%%%%%%%%%
\title[]{Survey research in the digital age}
\author[]{Bernhard Clemm von Hohenberg\\Department of Computational Social Science\\GESIS}
\date[]{Summer Institutes in Computational Social Science\\July 28, 2023}
\begin{document}
%%%%%%%%%%%%%%%%%%%%%%%%%%
\frame{\titlepage}
%%%%%%%%%%%%%%%%%%%%%%%%%%

\begin{frame}{Schedule}

\vspace{0.5em}
\begin{itemize}
\item 9.00-9.45 Introduction \& total error survey framework
\item 9.45-10.15 Probability and non-probability sampling
\vspace{0.5em}
\item Coffee break
\vspace{0.5em}
\item 10.30-11.00 Computer-administered interviewing
\item 11.00-11.30 Linking surveys to big data
\item \textcolor{violet}{11:30-13:00 Intro and begin group exercise}
\vspace{0.5em}
\item Lunch (or Eisbach plunge)
\vspace{0.5em}
\item 14:00-15:45 Continue group exercise
\end{itemize}

\end{frame}
%%%%%%%%%%%%%%%%%%%%%%%%%%%
\begin{frame}{Credits}

These materials build heavily on Matthew Salganik's 2019 SICSS class as well as Chapter 3 of ``Bit by Bit: Social Research in the Digital Age''.

\end{frame}
%%%%%%%%%%%%%%%%%%%%%%%%%%%
\begin{frame}{Motivating study}

\begin{center}
\includegraphics[width=0.9\textwidth]{figures/goel_online_2017_title}
\end{center}
\vfill
\TINY{\url{https://5harad.com/papers/dirtysurveys.pdf}}

\end{frame}
%%%%%%%%%%%%%%%%%%%%%%%%%%%
\begin{frame}{Activity}

\begin{itemize}
\item Design a questionnaire using questions already asked in high-quality, probability-based surveys (i.e., Eurobarometer)
\pause
\item Recruit participants from Prolific and have them complete your questionnaire  
\pause
\item Compare results from your survey to the results from Eurobarometer
\pause
\item Try different approaches to weighting and see how the change the estimates
\end{itemize}

\end{frame}
%%%%%%%%%%%%%%%%%%%%%%%%%%
\begin{frame}{Learnings}

This activity will give you practice:
\begin{itemize}
\item Designing questionnaires and working with survey software (Google Forms)
\pause
\item Collecting survey data on recruitment platforms (Prolific)
\pause
\item Pre-processing/analyzing survey survey data (R)
\pause
\item Applying post-stratification and thinking along total survey error framework
\pause
\end{itemize}

\vfill
Remember: This is a learning activity so try whatever you want.

\end{frame}
%%%%%%%%%%%%%%%%%%%%%%%%%%%
\begin{frame}{Workflow}

Recommended work flow (details in \nolinkurl{0_exercise-instructions.pdf}):

\begin{itemize}
\item Familiarize yourself with the Eurobarometer survey
\pause
\item Create survey on Google Forms 
\begin{itemize}
    \item Don't forget to collect the sociodemographics for post-stratification
    \item Measure multiple outcomes (estimates are also property of question not just sample)
\end{itemize}
\pause
\item Test questionnaire on your own devices (!)
\pause
\item Publish on Prolific with my help
\pause
\item Lunch break
\pause
\item Download data from Google Forms
\pause
\item Analyze your data and/or apply post-stratification with larger pre-simulated data 
\begin{itemize}
    \item To post-stratify, we will use Census data from UK Office of National Statistics
    \item We compare these post-stratification estimates to the Eurobarometer benchmarks
\end{itemize}
\end{itemize}

\end{frame}
%%%%%%%%%%%%%%%%%%%%%%%%%%
\begin{frame}

\begin{center}
\LARGE{A quick and dirty tour of post-stratification}
\end{center}

\end{frame}
%%%%%%%%%%%%%%%%%%%%%%%%
\begin{frame}{Post-stratification}

The principle is the following:
\begin{enumerate}
\item Chop up the sample into groups \pause
\item Estimate the mean in each group \pause
\item Combine the estimates for each group into an overall estimate
\end{enumerate}
\pause

\vfill

\begin{equation*}
\hat{\bar{y}}_{post} = \sum_{h=1}^H \frac{N_h}{N} \hat{\bar{y}}_h
\end{equation*}
where 
\begin{itemize}
\item $N$: size of the population
\item $N_h$: size of group $h$
\item $\hat{\bar{y}}_h$: estimated average outcome for group $h$
\end{itemize}

\end{frame}
%%%%%%%%%%%%%%%%%%%%%%%%
\begin{frame}{Poststratification}

Assumptions:
\begin{itemize}
\item The realized sample $s$ is partitioned into $H$ groups, $s_1, s_2, \ldots s_H$
\item Given $s$, all elements in $s_k$ are assumed to have the same response probability; different groups can have different response probabilities
\item Equivalent to data is missing completely at random (MCAR) within each group
\item ``Response Homogeneity Group Model'' (RHG Model), see Sarndal et al.\ (1992) Sec 15.6.2 (``A Useful Response Model'')
\end{itemize}

\vfill
If RHG model holds (and some other minor technical conditions), then the post-stratification estimator is unbiased. See Sarndal et al.\ (1992) Result 15.6.1 

\end{frame}
%%%%%%%%%%%%%%%%%%%%%%%%
\begin{frame}
\frametitle{Bias of cell-based post-stratification estimator}

If RHG does not hold and if the original sample is simple random sampling without replacement, then (Bethlehem, Cobben, and Schouten 2011, sec. 8.2.1):

$$bias(\hat{\bar{y}}_{post}) = \frac{1}{N} \sum_{h=1}^H \frac{cor(\phi_i, y_i)^{(h)} S(\phi_i)^{(h)} S(y_i)^{(h)}}{\bar{\phi}^{(h)}}$$

So, how should we create the $H$ groups? \pause
\begin{itemize}
\item form homogeneous groups where there is little variation in response propensity $(S(\phi_i)^{(h)} \approx 0)$ and the outcome $(S(y_i)^{(h)} \approx 0)$ \pause
\item form groups where the people that you see are like the people that you don't see $(cor(\phi_i, y_i)^{(h)} \approx 0)$
\end{itemize}

\vfill
In practice this can be difficult because you want to form many groups, but then you have noisy estimates for each group.

\end{frame}
%%%%%%%%%%%%%%%%%%%%%%%%
\begin{frame}{Post-stratification}

Note:
\begin{itemize}
\item Horvitz-Thompson estimation is individual-based weight
\item Post-stratification can better be understood as a group-based weight
\end{itemize}

\end{frame}
%%%%%%%%%%%%%%%%%%%%%%%%
\begin{frame}{Exercise plan}

Three increasingly sophisticated ways to make group estimate $\hat{\bar{y}}_h$. 
\begin{itemize}
\item cell-based post-stratification
\item model-based post-stratification
\item (Extra: multilevel regression post-stratification)
\end{itemize}

\end{frame}
%%%%%%%%%%%%%%%%%%%
\begin{frame}{Simple cell-based post-stratification}

For our example data, let's form 64 $(2 \times 2 \times 8)$ groups:
\begin{itemize}
\item sex (2 groups)
\item age (4 groups)
\item region (8 groups)
\end{itemize}

\begin{equation*}
\hat{\bar{y}}_h = \frac{\sum_{i \in h} y_i}{n_h}
\end{equation*}

\vfill
$h$ is a group described by a unique combination of gender (2 groups) $\times$ age (4 groups) $\times$ race (5 groups) $\times$ region (4 groups) 

\end{frame}
%%%%%%%%%%%%%%%%%%%%
\begin{frame}{Simple cell-based post-stratification}

\begin{itemize}
\item We can't make an estimate for each group. For example, we don't have any female, 65+ in the East of England \pause
\item This problem can arise if you have too many cell. We have a crude work-around (imputation) in the code. 
\end{itemize}

\end{frame}
%%%%%%%%%%%%%%
\begin{frame}{Model-based post-stratification}

\begin{equation*}
\hat{\bar{y}}_{post} = \sum_{h=1}^H \frac{N_h}{N} \hat{\bar{y}}_h
\end{equation*}
where
$\hat{\bar{y}}_h$ comes from an individual-level model

\begin{align*}
Pr(y_i = 1) = logit^{-1} (\beta_0 + \\
\beta_{male} \cdot male_i+ \\
\beta_{18to29} \cdot 18to29_i +
\beta_{30to49} \cdot 30to49_i + \beta_{50to64} \cdot 50to64_i+ \beta_{65plus} \cdot 65plus_i +\\
\beta_{NorthernIreland} \cdot NorthernIreland_i + \beta_{Wales} \cdot Wales_i ... + \beta_{London} \cdot London_i)
\end{align*}

\end{frame}

%%%%%%%%%%%%%%%%%%%%%%%%%%
\begin{frame}{Model-based post-stratification}

\begin{itemize}
\item Modeling allows you to make more estimates for smaller groups \pause
\item These techniques is widely used by modern pollsters (e.g., YouGov) and political scientists \pause
\end{itemize}

\end{frame}
%%%%%%%%%%%%%%%%%%%%%%%%%%
\begin{frame}

\begin{center}
You're unlikely to get through all of the activity---that's ok!
\end{center}

\end{frame}

\end{document}
