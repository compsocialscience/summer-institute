\documentclass[aspectratio=169]{beamer}
\usepackage{color,amsmath}
\usepackage{subfigure}
\usepackage{booktabs}
\usepackage{framed}
\usepackage{comment}

\usepackage{ulem}

\usepackage{hyperref}
\hypersetup{
    colorlinks=true,
    linkcolor=blue,
    filecolor=magenta,      
    urlcolor=cyan,
}

%%%%%%%%%%%%%%%%%%%%%%%%%%
\title[]{Day 4: Survey research in the digital age}
\author[]{Matthew J. Salganik\\Department of Sociology\\Princeton University}
\date[]{%Summer Institutes in Computational Social Science\\2020
%\vfill
%\begin{flushleft}
%{\scriptsize
%The Summer Institutes in Computational Social Science is supported by grants from the Russell Sage Foundation and the Alfred P. Sloan Foundation.}
%\end{flushleft}
\begin{flushright}
\vfill
\includegraphics[width=0.1\textwidth]{figures/cc-by.png}
\end{flushright}
}
\begin{document}
%%%%%%%%%%%%%%%%%%%%%%%%%%
\frame{\titlepage}
%%%%%%%%%%%%%%%%%%%%%%%%%%
\begin{frame}

\begin{center}
\Large{Feedback on the feedback}
\end{center}

\end{frame}
%%%%%%%%%%%%%%%%%%%%%%%%%%
\begin{frame}

\begin{itemize}
\item Collaboration \pause
\item Collaboration with people you don't know \pause
\item Collaboration with people you don't know from other fields \pause
\item Collaboration with people you don't know from other fields remotely \pause
\end{itemize}

\vfill
Step back and step up

\end{frame}
%%%%%%%%%%%%%%%%%%%%%%%%%%
\begin{frame}

This activity has three chunks:
\begin{itemize}
\item Build the questionnaire \pause
\item Deploy to MTurk and pay workers (be generous to workers) \pause
\item Post-stratification \pause
\end{itemize}

Groups have mixed expertise and are 3 people so everyone takes on turn in the driver seat
\begin{itemize}
\item challenge yourself
\item teach others
\end{itemize}

\end{frame}
%%%%%%%%%%%%%%%%%%%%%%%%%%
\begin{frame}

\begin{center}
\Large{Robin has some ideas about new approaches to group coding}
\end{center}

\end{frame}
%%%%%%%%%%%%%%%%%%%%%%%%%%
\begin{frame}

\begin{center}
\Large{Let's make some lunch topics}
\end{center}

\end{frame}
%%%%%%%%%%%%%%%%%%%%%%%%%%
\begin{frame}

\url{https://compsocialscience.github.io/summer-institute/2020/duke/schedule}

\vfill
A bit more time in the big group and a bit less time in the small groups

\end{frame}
%%%%%%%%%%%%%%%%%%%%%%%%%%
\begin{frame}

\begin{center}
\begin{tabular}{ccc}
\onslide<1-3>\includegraphics[width=0.30\textwidth]{figures/duchamp_fountain} & \phantom{12345} & \onslide<2-3>{\includegraphics[width=0.30\textwidth]{figures/michelangelo_david}} \\
\onslide<3>{\LARGE{readymades}} &  & \onslide<3>{\LARGE{custommades}}
\end{tabular}
\end{center}

\vfill
\onslide<3>{
\TINY{\url{https://commons.wikimedia.org/wiki/File:Duchamp_Fountaine.jpg}}\\
\TINY{\url{https://commons.wikimedia.org/wiki/File:\%27David\%27_by_Michelangelo_JBU0001.JPG}}}

\end{frame}
%%%%%%%%%%%%%%%%%%%%%%%%%%
\begin{frame}
\frametitle{Learning objectives}

Participants will gain experience with the following activities:
\begin{itemize}
\item reading survey results and methodology reports
\item creating questionnaires on Google Forms
\item deploying surveys on MTurk
\item data wrangling and weighing non-probability samples
\item using the total survey error framework to reason about and discussion sources of errors in estimates
\end{itemize}

\end{frame}
%%%%%%%%%%%%%%%%%%%%%%%%%%
\begin{frame}

By 11:30 you should create your questionnaire and deploy to MTurk . . . 

\end{frame}
%%%%%%%%%%%%%%%%%%%%%%%%%%

\end{document}
