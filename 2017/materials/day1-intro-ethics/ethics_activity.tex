\documentclass{article}
%\usepackage{fullpage}
\usepackage{graphicx, comment, color, url}
\usepackage[round]{natbib} 
\usepackage{booktabs}
\usepackage{subfigure}
%\usepackage[small,compact]{titlesec} % saves space
\usepackage[margin=1in]{geometry}

\title{Ethics activity\\Summer Institute in Computational Social Science\footnotemark[1]}
\author{Matthew J. Salganik and Yo-Yo Chen}
\date{June 19, 2017}

\begin{document}
\maketitle
\renewcommand{\thefootnote}{\fnsymbol{footnote}}
\thispagestyle{empty}
\footnotetext[1]{We thank Don Green for advice that helped improve this activity.  This activity is based on a similar activity in Chapter 6 of~\citet{salganik_bit_2017}.}
\renewcommand{\thefootnote}{\arabic{footnote}}

In August 2006, about 10 days prior to the primary election, 20,000 people living in Michigan received a mailing that showed their voting behavior and the voting behavior of their neighbors (Fig.~\ref{fig:gerber_social_2008_neighbors}). One piece mailings typically increase voter turnout by about one percentage point, but this one increased turnout by 8.1 percentage points, the largest effect seen up to that point~\citep{gerber_social_2008}.  The effect was so large that a political operative named Hal Malchow offered Donald Green \$100,000 not to publish the result of the experiment, presumably so that Malchow could make use of this information himself~\citep[p. 304]{issenberg_victory_2012}.  But, Alan Gerber, Donald Green, and Christopher Larimer did publish the paper in 2008 in the \textit{American Political Science Review.}  

When you carefully inspect the mailer in Gerber et al. (2008) you may notice that the researchers' names do not appear on it.  Rather, the return address is to Practical Political Consulting.  In the acknowledgment to the paper the authors explain: ``Special thanks go to Mark Grebner of Practical Political Consulting, who designed and administered the mail program studied here.''
 
In addition to these 20,000 mailers, the researchers also sent 60,000 other potentially less sensitive mailers (Fig.~\ref{fig:gerber_social_2008_civicduty},~\ref{fig:gerber_social_2008_hawthorne}, and \ref{fig:gerber_social_2008_self}) Then, there was a backlash from participants.  In fact,~\citet[p. 198]{issenberg_victory_2012} reports that ``Grebner [the director of Practical Political Consulting] was never able to calculate how many people took the trouble to complain by phone, because his office answering machine filled so quickly that new callers were unable to leave a message.'' In fact, Grebner noted that the backlash could have been even larger if they had scaled up the treatment.  He said to Alan Gerber, one of the researchers, ``Alan if we had spent five hundred thousand dollars and covered the whole state you and I would be living with Salman Rushdie.''~\citep[p. 200]{issenberg_victory_2012}.
 
This particular case is great for a discussion because it is interesting and important research, and it touches on many of the ethical issues that arise in computational social science.  But it touches on these issues in unexpected ways, and so it can help us move beyond some stale debates that have been happening for a while.  It is also a great case because there are no easy answers.  Please discuss these question in your group:

\begin{enumerate}
\item Assess the ethical issues raised by this study.  Please draw on any framework, principles, or ideas that you think are appropriate.
\item Given your assessment, what approaches would you take to address the ethical issues related to this study?  These approaches could be related to the design, testing, or publishing of the study.
\item Would it impact your answer to the questions above if Mark Grebner was already sending out similar mailings at this time?  More generally, how should researchers think about evaluating existing interventions created by practitioners?
\item How would your decisions in (2) impact the research objective (the ability to understand the cause of a difference in voting rates between the treatment and control group)?
\end{enumerate}
 
Prepare a 5 minute presentation for us (people who are already familiar with this case and the ethical concepts that we have learned).  All groups will present and then we will have a group discussion about similarities and differences in people's approaches.
 
If you have extra time: Ethical response surveys are a technique for collecting reactions to proposed study designs~\citep{schechter_using_2014}.  Building on what you did above, design an ethical response survey to compare three approaches to this style of experiment.
 
If you have even more extra time: Actually conduct the ethical response survey using participants on Amazon Mechanical Turk.


\begin{figure}
\centering
\includegraphics[width=\textwidth]{figures/gerber_social_2008_neighbors}
\caption{Neighbor mailer from~\citet{gerber_social_2008}.}
\label{fig:gerber_social_2008_neighbors}
\end{figure}

\begin{figure}
\centering
\includegraphics[width=\textwidth]{figures/gerber_social_2008_civicduty}
\caption{Civic duty mailer from~\citet{gerber_social_2008}.}
\label{fig:gerber_social_2008_civicduty}
\end{figure}

\begin{figure}
\centering
\includegraphics[width=\textwidth]{figures/gerber_social_2008_hawthorne}
\caption{Hawthorne mailer from~\citet{gerber_social_2008}.}
\label{fig:gerber_social_2008_hawthorne}
\end{figure}

\begin{figure}
\centering
\includegraphics[width=\textwidth]{figures/gerber_social_2008_self}
\caption{Self mailer from~\citet{gerber_social_2008}.}
\label{fig:gerber_social_2008_self}
\end{figure}

\section*{Schedule} 
 
\begin{itemize}
\item About 1:30: Begin
\item 2:30 - 3:30: Presentation
\item 3:30 - 4:00: Discussion and wrap-up
\end{itemize} 
 
 
 
 
\bibliographystyle{apalike} 
\bibliography{bitbybit}
\end{document}
